\documentclass[11pt]{report}
\usepackage[utf8]{inputenc}
\usepackage[a4paper, margin=1in]{geometry}
\usepackage{graphicx, color}
\usepackage{booktabs}
\usepackage[toc,page]{appendix}
\usepackage{pdfpages}
\usepackage{multicol}
\usepackage{changepage}
\usepackage{float}
\usepackage{multirow}
\usepackage{amsmath}
\usepackage[encoding,filenameencoding=utf8]{grffile}
\usepackage{subcaption}
\usepackage{csquotes}
\usepackage{hyperref, bookmark}

\setlength{\columnseprule}{0.5pt}
\def\columnseprulecolor{\color{black}}


% Package de algoritmos
\usepackage[portuguese,ruled,lined]{algorithm2e}

% Portuguese Encoding 
\usepackage[portuguese]{babel}
\usepackage[T1]{fontenc}

% import Java
\usepackage{listings} 

% Header
\usepackage{fancyhdr}
\pagestyle{fancy}
	\fancyhf{}
		\lhead{LEIC - ISEL}
    \rhead{Comunicações}

% Footer
\renewcommand{\headrulewidth}{1pt}
\renewcommand{\footrulewidth}{1pt}
\fancyfoot[CE,CO]{\thepage}


%%%%%%%%%%%%%%%%%%%%%%%%%%%%%%%%%%%%%%%%% CHANGES %%%%%%%%%%%%%%%%%%%%%%%%%%%%%%%%%


% Redefine the plain page style
\fancypagestyle{plain}{
\pagestyle{fancy}
  \fancyhf{}
	\lhead{LEIC - ISEL}
	\rhead{Segurança Informática}
	\renewcommand{\headrulewidth}{1pt}
	\renewcommand{\footrulewidth}{1pt}
\fancyfoot[CE,CO]{\thepage}
}

%%%%%%%%%%%%%%%%%%%%%%%%%%%%%%%%%%%%%%%%% CHANGES %%%%%%%%%%%%%%%%%%%%%%%%%%%%%%%%%

% Titulo e informação de capa
\title{Primeira Serie de exercícios}
\date{Outubro 2019}

% Define Chapter heading
\makeatletter
\def\@makechapterhead#1{%
  \vspace*{20\p@}%                                 % Insert 50pt (vertical) space
  {\parindent \z@ \raggedright \normalfont         % No paragraph indent, ragged right
%    \interlinepenalty\@M                           % Penalty
    \huge \bfseries #1\par\nobreak                 % Huge, bold chapter title
    \vskip 10\p@                                   % Insert 40pt (vertical) space
  }}
  
  %%%%%%%%%%%%%%%%%%%%%%%%%%%%%%%%%%%%%%%%% CHANGES %%%%%%%%%%%%%%%%%%%%%%%%%%%%%%%%%

\makeatother

%Custom commands

\renewcommand\appendixpagename{Anexos}
\renewcommand\appendixname{Anexo}
\renewcommand\appendixtocname{Anexos}

\newenvironment{subs}
  {\adjustwidth{2.5em}{0pt}}
  {\endadjustwidth}


% Retira a indentação ao início dos parágrafos 
\setlength{\parindent}{0em}
\newcommand{\blank}[1]{\hspace*{#1}\linebreak[0]}

% Section and Subsection spacing:
\usepackage{titlesec}
\usepackage{lipsum}



%%%%%%%%%%%%%%%%%%%%%%%%%%%%%%%%%%	DOCUMENT BEGINS	%%%%%%%%%%%%%%%%%%%%%%%%%%%%%%%%%%%%%

\begin{document}

%%%%%%%%%%%%%%%%%%%%%%%%%%%%%%%%%%%%%%%%%%%%%%%%%%%%%%%%%%%%%%%%%%%%%%%%%%%%%%%%%%%%%%%%%
%										COVER											%
%%%%%%%%%%%%%%%%%%%%%%%%%%%%%%%%%%%%%%%%%%%%%%%%%%%%%%%%%%%%%%%%%%%%%%%%%%%%%%%%%%%%%%%%%

\begin{titlepage} 

\newcommand{\HRule}{\rule{\linewidth}{0.5mm}} % Defines a new command for the horizontal lines, change thickness here

%----------------------------------------------------------------------------------------
%	LOGO SECTION
%----------------------------------------------------------------------------------------

\includegraphics[width=130pt, keepaspectratio=true]{img/logo_isel}\\[1cm] % Include a department/university logo - this will require the graphicx package
\center % Center everything on the page
 
%----------------------------------------------------------------------------------------
%	HEADING SECTIONS
%----------------------------------------------------------------------------------------

\textsc{\LARGE INSTITUTO SUPERIOR DE ENGENHARIA DE LISBOA}\\[1.5cm] % Name of your university/college
\vskip 40pt
\textsc{\Large SEGURANÇA INFORMÁTICA}\\[0.5cm] % Major heading such as course name
\textsc{\large LICENCIATURA ENGENHARIA INFORMÁTICA E COMPUTADORES}\\[0.5cm] % Minor heading such as course title
\vskip 40pt

%----------------------------------------------------------------------------------------
%	TITLE SECTION
%----------------------------------------------------------------------------------------

\HRule \\[0.4cm]
{ \LARGE \bfseries FASE DE EXERCÍCIOS }\\[0.4cm]
{ \huge \bfseries Fase 1}\\[0.4cm] % Title of your document
\HRule \\[1.5cm]
 
%----------------------------------------------------------------------------------------
%	AUTHOR SECTION
%----------------------------------------------------------------------------------------
\vskip 70pt
\begin{minipage}{0.4\textwidth}
\begin{flushleft} \large
\emph{Autores:}\\
43552 - Samuel \textsc{Costa}\\
43320 - André \textsc{Mendes}
\end{flushleft}
\end{minipage}
~
\begin{minipage}{0.4\textwidth}
\begin{flushright} \large
\emph{Docente:} \\
José \textsc{Simão}\\
\end{flushright}
\end{minipage}\\[3cm]

%----------------------------------------------------------------------------------------
%	DATE SECTION
%----------------------------------------------------------------------------------------

{\large 29 de Dezembro de 2018}\\[3cm] % Date, change the \today to a set date if you want to be precise

%----------------------------------------------------------------------------------------

\vfill % Fill the rest of the page with whitespace

\end{titlepage}


\renewcommand\thesection{\arabic{section}}

\pdfbookmark{\contentsname}{toc}
\tableofcontents


\newpage


\section{Introdução}
O trabalho realizado para este fase pretende que os temas desenvolvidos durante as aulas sejam postos em prática. Para esta fase os exercícios focaram essencialmente a primeira parte da matéria.\\

\begin{itemize}
  \item Esquemas e Primitivas Criptográficas
  \item Java Cryptographic Architecture (JCA)
  \item Certificados digitais e Infraestrutura de Chave Pública
  \item Protocolo Criptográfico Transport Layer Security (SSL/TLS)
\end{itemize}

Neste trabalho prático pretendemos responder aos vários exercícios propostos e implementar uma demonstração dos pontos referidos.

\newpage

\section{Primeira série de exercícios}
\subsection{Exercício 1}

Ambos os esquemas criptográficos visam garantir a autenticidade da mensagem adicionando à própria mensagem um numero de bits de forma a identificar o emissor da mesma, no caso do MAC uma marca ou uma assinatura no caso da assinatura digital, sendo essa autenticidade verificada no recetor. Ambos os esquemas utilizam chaves privadas para "marcar" \space a mensagem, sendo que no caso da assinatura digital é usado uma chave publica para autenticar essa mesma mensagem, já no esquema MAC é usada a chave privada do emissor.\\

\subsection{Exercício 2}
A função de hash obriga a uma mensagem de tamanho variável $ m $ para uma redução da mesma de tamanho $ n $ ou seja, a dimensão do $ yL = $ dimensão de $ m1 $, logo $ n $ bits $ = $ dimensão de um bloco "shiftado"\space da dimensão da mensagem, pelo que existe $ m $ e $ m' $, tal que $ H(m) = H(m')$.\\

\subsection{Exercício 3}
Um algoritmo de cifra determinístico indica que a cifra de uma mensagem pode se repetir, ou seja, caso a mensagem seja um padrão a cifra irá se repetir também, logo o atacante tendo acesso à cifra e à chave publica pode através de força bruta repetir uma mensagem até encontrar a cifra, pois se for igual aquela que o atacante então a mensagem é equivalente à original.\\

\subsection{Exercício 4}
Analisando os dois tipos de esquemas seria expectável que a primitiva AES é sempre mais dificil de criptoanalizar que um sistema com primitiva DES, no entanto essa analise tem de ser feita também relativamente ao modo de operação que cada uma pode usar, como por exemplo no caso especifico de um sistema AES ter usado um modo de operação ECB (Electronic Code Book) e a sua mensagem ser de alguma forma curta e padronizada pode mais facilmente ser quebrada que uma DES que usou o mode de operação CBC (Cipher Block Chain).

\newpage



\section{Conclusão}


\end{document}
