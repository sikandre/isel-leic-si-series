\documentclass[11pt]{report}
\usepackage[utf8]{inputenc}
\usepackage[a4paper, margin=1in]{geometry}
\usepackage{graphicx, color}
\usepackage{booktabs}
\usepackage[toc,page]{appendix}
\usepackage{pdfpages}
\usepackage{multicol}
\usepackage{changepage}
\usepackage{float}
\usepackage{multirow}
\usepackage{amsmath}
\usepackage[encoding,filenameencoding=utf8]{grffile}
\usepackage{subcaption}
\usepackage{csquotes}
\usepackage{hyperref, bookmark}

\setlength{\columnseprule}{0.5pt}
\def\columnseprulecolor{\color{black}}


% Package de algoritmos
\usepackage[portuguese,ruled,lined]{algorithm2e}

% Portuguese Encoding 
\usepackage[portuguese]{babel}
\usepackage[T1]{fontenc}

% import Java
\usepackage{listings} 

% Header
\usepackage{fancyhdr}
\pagestyle{fancy}
	\fancyhf{}
		\lhead{LEIC - ISEL}
    \rhead{Segurança Informática}

% Footer
\renewcommand{\headrulewidth}{1pt}
\renewcommand{\footrulewidth}{1pt}
\fancyfoot[CE,CO]{\thepage}


%%%%%%%%%%%%%%%%%%%%%%%%%%%%%%%%%%%%%%%%% CHANGES %%%%%%%%%%%%%%%%%%%%%%%%%%%%%%%%%


% Redefine the plain page style
\fancypagestyle{plain}{
\pagestyle{fancy}
  \fancyhf{}
	\lhead{LEIC - ISEL}
	\rhead{Segurança Informática}
	\renewcommand{\headrulewidth}{1pt}
	\renewcommand{\footrulewidth}{1pt}
\fancyfoot[CE,CO]{\thepage}
}

%%%%%%%%%%%%%%%%%%%%%%%%%%%%%%%%%%%%%%%%% CHANGES %%%%%%%%%%%%%%%%%%%%%%%%%%%%%%%%%

% Titulo e informação de capa
\title{Primeira Serie de exercícios}
\date{Outubro 2019}

% Define Chapter heading
\makeatletter
\def\@makechapterhead#1{%
  \vspace*{20\p@}%                                 % Insert 50pt (vertical) space
  {\parindent \z@ \raggedright \normalfont         % No paragraph indent, ragged right
%    \interlinepenalty\@M                           % Penalty
    \huge \bfseries #1\par\nobreak                 % Huge, bold chapter title
    \vskip 10\p@                                   % Insert 40pt (vertical) space
  }}
  
  %%%%%%%%%%%%%%%%%%%%%%%%%%%%%%%%%%%%%%%%% CHANGES %%%%%%%%%%%%%%%%%%%%%%%%%%%%%%%%%

\makeatother

%Custom commands

\renewcommand\appendixpagename{Anexos}
\renewcommand\appendixname{Anexo}
\renewcommand\appendixtocname{Anexos}

\newenvironment{subs}
  {\adjustwidth{2.5em}{0pt}}
  {\endadjustwidth}


% Retira a indentação ao início dos parágrafos 
\setlength{\parindent}{0em}
\newcommand{\blank}[1]{\hspace*{#1}\linebreak[0]}

% Section and Subsection spacing:
\usepackage{titlesec}
\usepackage{lipsum}



%%%%%%%%%%%%%%%%%%%%%%%%%%%%%%%%%%	DOCUMENT BEGINS	%%%%%%%%%%%%%%%%%%%%%%%%%%%%%%%%%%%%%

\begin{document}

%%%%%%%%%%%%%%%%%%%%%%%%%%%%%%%%%%%%%%%%%%%%%%%%%%%%%%%%%%%%%%%%%%%%%%%%%%%%%%%%%%%%%%%%%
%										COVER											%
%%%%%%%%%%%%%%%%%%%%%%%%%%%%%%%%%%%%%%%%%%%%%%%%%%%%%%%%%%%%%%%%%%%%%%%%%%%%%%%%%%%%%%%%%

\begin{titlepage} 

\newcommand{\HRule}{\rule{\linewidth}{0.5mm}} % Defines a new command for the horizontal lines, change thickness here

%----------------------------------------------------------------------------------------
%	LOGO SECTION
%----------------------------------------------------------------------------------------

\includegraphics[width=130pt, keepaspectratio=true]{img/logo_isel}\\[1cm] % Include a department/university logo - this will require the graphicx package
\center % Center everything on the page
 
%----------------------------------------------------------------------------------------
%	HEADING SECTIONS
%----------------------------------------------------------------------------------------

\textsc{\LARGE INSTITUTO SUPERIOR DE ENGENHARIA DE LISBOA}\\[1.5cm] % Name of your university/college
\vskip 40pt
\textsc{\Large SEGURANÇA INFORMÁTICA}\\[0.5cm] % Major heading such as course name
\textsc{\large LICENCIATURA ENGENHARIA INFORMÁTICA E COMPUTADORES}\\[0.5cm] % Minor heading such as course title
\vskip 40pt

%----------------------------------------------------------------------------------------
%	TITLE SECTION
%----------------------------------------------------------------------------------------

\HRule \\[0.4cm]
{ \LARGE \bfseries FASE DE EXERCÍCIOS }\\[0.4cm]
{ \huge \bfseries Fase 1}\\[0.4cm] % Title of your document
\HRule \\[1.5cm]
 
%----------------------------------------------------------------------------------------
%	AUTHOR SECTION
%----------------------------------------------------------------------------------------
\vskip 70pt
\begin{minipage}{0.4\textwidth}
\begin{flushleft} \large
\emph{Autores:}\\
43552 - Samuel \textsc{Costa}\\
43320 - André \textsc{Mendes}
\end{flushleft}
\end{minipage}
~
\begin{minipage}{0.4\textwidth}
\begin{flushright} \large
\emph{Docente:} \\
José \textsc{Simão}\\
\end{flushright}
\end{minipage}\\[3cm]

%----------------------------------------------------------------------------------------
%	DATE SECTION
%----------------------------------------------------------------------------------------

{\large 25 de Novembro de 2019}\\[3cm] % Date, change the \today to a set date if you want to be precise

%----------------------------------------------------------------------------------------

\vfill % Fill the rest of the page with whitespace

\end{titlepage}


\renewcommand\thesection{\arabic{section}}

\pdfbookmark{\contentsname}{toc}
\tableofcontents


\newpage


\section*{Introdução}
O trabalho realizado para este fase pretende que os temas desenvolvidos durante as aulas sejam postos em prática. Para esta fase os exercícios focaram essencialmente a segunda parte da matéria.\\

\begin{itemize}
  \item Autenticação baseadas em passwords
  \item Gestão de Identidade em Aplicações Web
  \item Modelos de Controlo de Acesso
\end{itemize}

Neste trabalho prático pretendemos responder aos vários exercícios propostos e implementar uma demonstração dos pontos referidos.

\newpage


\section{Exercício 1}
	\subsection*{1.1}
	Um esquema de assinatura as chaves usadas são assimétricas o que permite a que autentica a mensagem usa a sua chave privada para o efeito e o recetor usa a chave publica do emissor para a autenticar essa mesma mensagem. Um esquema Mac como a chave é simétrica numa fase inicial através de um canal seguro teriam que ser trocadas para que desta forma ambos os intervenientes tenham a mesma chave para verificar a autenticidade da mensagem.
	
	\subsection*{1.2}
	As recomendações do texto abordam fragilidades de segurança no caso de um atacante conseguir obter a chave privada de um servidor durante a sessão estabelecida entre o cliente e o servidor. No caso de um atacante obter a chave privada de um servidor passa a poder obter todas as mensagens dirigidas a ele, mas com a nova diretiva obriga a que a cada sessão durante a fase de \textit{handshake} sejam estabelecidas novas chaves o que faz com que a chave do atacante deixe ter utilidade.

\section{Exercício 2}
	O algoritmo simétrico para cifrar a \textit{password} de um utilizador é criado através da função de \textit{hash} com um valor aleatório unico a cada utilizador prevenindo assim que ataques de dicionario, ou seja, \textit{passwords} iguais passam a ter valores de \textit{hash} diferentes tornando a sua chave unica para cada utilizador. Para realizar a decifra realiza se o mesmo processo à \textit{password} introduzida pelo utilizador, usa se o mesmo valor de \textit{salt} e realiza se a comparação devolvendo um verdadeiro ou falso.

\section{Exercício 3}
	A criação de um \textit{cookie} passa por vaias fases até ser guardado no lado do cliente para garantir a autenticidade durante a sua sessão sem que este tenha que validar as suas credencias constantemente.
	O processo passa pelos seguintes passos:
	\begin{enumerate}
		\item O utilizador garante a sua autenticidade através de credenciais introduzidas no lado do servidor ou através de um redirecionamento para um fornecedor de identidade.
		\item O utilizador é então autenticado recebendo um código. 
		\item É novamente redirecionado para o servidor sendo que este troca o código por um \textit{id\_token}.
		\item O servidor realiza a operação de \textit{setCookie} por forma a que o utilizador nas proximas chamadas envie o \textit{cookie} para que este não tenha a necessidade de se autenticar.
		\item O servidor autenticou o \textit{cookie} e de todas as vezes realizar uma operação de autenticação para o confirmar.
	\end{enumerate}

\section{Exercício 4}
	\subsection*{4.1}
	
	
	\subsection*{4.2}
	
	
	\subsection*{4.3}

\section{Exercício 5}


\newpage


\end{document}
