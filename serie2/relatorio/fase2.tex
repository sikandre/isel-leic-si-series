\documentclass[11pt]{report}
\usepackage[utf8]{inputenc}
\usepackage[a4paper, margin=1in]{geometry}
\usepackage{graphicx, color}
\usepackage{booktabs}
\usepackage[toc,page]{appendix}
\usepackage{pdfpages}
\usepackage{multicol}
\usepackage{changepage}
\usepackage{float}
\usepackage{multirow}
\usepackage{amsmath}
\usepackage[encoding,filenameencoding=utf8]{grffile}
\usepackage{subcaption}
\usepackage{csquotes}
\usepackage{hyperref, bookmark}

\setlength{\columnseprule}{0.5pt}
\def\columnseprulecolor{\color{black}}


% Package de algoritmos
\usepackage[portuguese,ruled,lined]{algorithm2e}

% Portuguese Encoding 
\usepackage[portuguese]{babel}
\usepackage[T1]{fontenc}

% import Java
\usepackage{listings} 

% Header
\usepackage{fancyhdr}
\pagestyle{fancy}
	\fancyhf{}
		\lhead{LEIC - ISEL}
    \rhead{Segurança Informática}

% Footer
\renewcommand{\headrulewidth}{1pt}
\renewcommand{\footrulewidth}{1pt}
\fancyfoot[CE,CO]{\thepage}


%%%%%%%%%%%%%%%%%%%%%%%%%%%%%%%%%%%%%%%%% CHANGES %%%%%%%%%%%%%%%%%%%%%%%%%%%%%%%%%


% Redefine the plain page style
\fancypagestyle{plain}{
\pagestyle{fancy}
  \fancyhf{}
	\lhead{LEIC - ISEL}
	\rhead{Segurança Informática}
	\renewcommand{\headrulewidth}{1pt}
	\renewcommand{\footrulewidth}{1pt}
\fancyfoot[CE,CO]{\thepage}
}

%%%%%%%%%%%%%%%%%%%%%%%%%%%%%%%%%%%%%%%%% CHANGES %%%%%%%%%%%%%%%%%%%%%%%%%%%%%%%%%

% Titulo e informação de capa
\title{Primeira Serie de exercícios}
\date{Outubro 2019}

% Define Chapter heading
\makeatletter
\def\@makechapterhead#1{%
  \vspace*{20\p@}%                                 % Insert 50pt (vertical) space
  {\parindent \z@ \raggedright \normalfont         % No paragraph indent, ragged right
%    \interlinepenalty\@M                           % Penalty
    \huge \bfseries #1\par\nobreak                 % Huge, bold chapter title
    \vskip 10\p@                                   % Insert 40pt (vertical) space
  }}
  
  %%%%%%%%%%%%%%%%%%%%%%%%%%%%%%%%%%%%%%%%% CHANGES %%%%%%%%%%%%%%%%%%%%%%%%%%%%%%%%%

\makeatother

%Custom commands

\renewcommand\appendixpagename{Anexos}
\renewcommand\appendixname{Anexo}
\renewcommand\appendixtocname{Anexos}

\newenvironment{subs}
  {\adjustwidth{2.5em}{0pt}}
  {\endadjustwidth}


% Retira a indentação ao início dos parágrafos 
\setlength{\parindent}{0em}
\newcommand{\blank}[1]{\hspace*{#1}\linebreak[0]}

% Section and Subsection spacing:
\usepackage{titlesec}
\usepackage{lipsum}



%%%%%%%%%%%%%%%%%%%%%%%%%%%%%%%%%%	DOCUMENT BEGINS	%%%%%%%%%%%%%%%%%%%%%%%%%%%%%%%%%%%%%

\begin{document}

%%%%%%%%%%%%%%%%%%%%%%%%%%%%%%%%%%%%%%%%%%%%%%%%%%%%%%%%%%%%%%%%%%%%%%%%%%%%%%%%%%%%%%%%%
%										COVER											%
%%%%%%%%%%%%%%%%%%%%%%%%%%%%%%%%%%%%%%%%%%%%%%%%%%%%%%%%%%%%%%%%%%%%%%%%%%%%%%%%%%%%%%%%%

\begin{titlepage} 

\newcommand{\HRule}{\rule{\linewidth}{0.5mm}} % Defines a new command for the horizontal lines, change thickness here

%----------------------------------------------------------------------------------------
%	LOGO SECTION
%----------------------------------------------------------------------------------------

\includegraphics[width=130pt, keepaspectratio=true]{img/logo_isel}\\[1cm] % Include a department/university logo - this will require the graphicx package
\center % Center everything on the page
 
%----------------------------------------------------------------------------------------
%	HEADING SECTIONS
%----------------------------------------------------------------------------------------

\textsc{\LARGE INSTITUTO SUPERIOR DE ENGENHARIA DE LISBOA}\\[1.5cm] % Name of your university/college
\vskip 40pt
\textsc{\Large SEGURANÇA INFORMÁTICA}\\[0.5cm] % Major heading such as course name
\textsc{\large LICENCIATURA ENGENHARIA INFORMÁTICA E COMPUTADORES}\\[0.5cm] % Minor heading such as course title
\vskip 40pt

%----------------------------------------------------------------------------------------
%	TITLE SECTION
%----------------------------------------------------------------------------------------

\HRule \\[0.4cm]
{ \LARGE \bfseries FASE DE EXERCÍCIOS }\\[0.4cm]
{ \huge \bfseries Fase 1}\\[0.4cm] % Title of your document
\HRule \\[1.5cm]
 
%----------------------------------------------------------------------------------------
%	AUTHOR SECTION
%----------------------------------------------------------------------------------------
\vskip 70pt
\begin{minipage}{0.4\textwidth}
\begin{flushleft} \large
\emph{Autores:}\\
43552 - Samuel \textsc{Costa}\\
43320 - André \textsc{Mendes}
\end{flushleft}
\end{minipage}
~
\begin{minipage}{0.4\textwidth}
\begin{flushright} \large
\emph{Docente:} \\
José \textsc{Simão}\\
\end{flushright}
\end{minipage}\\[3cm]

%----------------------------------------------------------------------------------------
%	DATE SECTION
%----------------------------------------------------------------------------------------

{\large 25 de Novembro de 2019}\\[3cm] % Date, change the \today to a set date if you want to be precise

%----------------------------------------------------------------------------------------

\vfill % Fill the rest of the page with whitespace

\end{titlepage}


\renewcommand\thesection{\arabic{section}}

\pdfbookmark{\contentsname}{toc}
\tableofcontents


\newpage


\section*{Introdução}
O trabalho realizado para este fase pretende que os temas desenvolvidos durante as aulas sejam postos em prática. Para esta fase os exercícios focaram essencialmente a segunda parte da matéria.\\

\begin{itemize}
  \item Autenticação baseadas em passwords
  \item Gestão de Identidade em Aplicações Web
  \item Modelos de Controlo de Acesso
\end{itemize}

Neste trabalho prático pretendemos responder aos vários exercícios propostos e implementar uma demonstração dos pontos referidos.

\newpage


\section{Exercício 1}
	\subsection*{1.1}
	Um esquema de assinatura as chaves usadas são assimétricas o que permite a que autentica a mensagem usa a sua chave privada para o efeito e o recetor usa a chave publica do emissor para a autenticar essa mesma mensagem. Um esquema Mac como a chave é simétrica numa fase inicial através de um canal seguro teriam que ser trocadas para que desta forma ambos os intervenientes tenham a mesma chave para verificar a autenticidade da mensagem.
	
	\subsection*{1.2}
	As recomendações do texto abordam fragilidades de segurança no caso de um atacante conseguir obter a chave privada de um servidor durante a sessão estabelecida entre o cliente e o servidor. No caso de um atacante obter a chave privada de um servidor passa a poder obter todas as mensagens dirigidas a ele, mas com a nova diretiva obriga a que a cada sessão durante a fase de \textit{handshake} sejam estabelecidas novas chaves o que faz com que a chave do atacante deixe ter utilidade.

\section{Exercício 2}
	O algoritmo simétrico para cifrar a \textit{password} de um utilizador é criado através da função de \textit{hash} com um valor aleatório unico a cada utilizador prevenindo assim que ataques de dicionario, ou seja, \textit{passwords} iguais passam a ter valores de \textit{hash} diferentes tornando a sua chave unica para cada utilizador. Para realizar a decifra realiza se o mesmo processo à \textit{password} introduzida pelo utilizador, usa se o mesmo valor de \textit{salt} e realiza se a comparação devolvendo um verdadeiro ou falso.

\section{Exercício 3}
	A criação de um \textit{cookie} passa por vaias fases até ser guardado no lado do cliente para garantir a autenticidade durante a sua sessão sem que este tenha que validar as suas credencias constantemente.
	O processo passa pelos seguintes passos:
	\begin{enumerate}
		\item O utilizador garante a sua autenticidade através de credenciais introduzidas no lado do servidor ou através de um redirecionamento para um fornecedor de identidade.
		\item O utilizador é então autenticado recebendo um código. 
		\item É novamente redirecionado para o servidor sendo que este troca o código por um \textit{id\_token}.
		\item O servidor realiza a operação de \textit{setCookie} por forma a que o utilizador nas proximas chamadas envie o \textit{cookie} para que este não tenha a necessidade de se autenticar.
		\item O servidor autenticou o \textit{cookie} e de todas as vezes realizar uma operação de autenticação para o confirmar.
	\end{enumerate}

\newpage
\section{Exercício 4}
	\subsection*{4.1}
	O parâmetro \textit{scope} refere-se ao access token a ser gerado e enumera o acesso que a aplicação cliente pretende ter.
	\subsection*{4.2}
	O client id e o client secret servem para o relying party se autenticar junto do fornecedor de identidade, e, como tal, é usado num pedido POST originado no relying party para o fornecedor de identidade. Portanto, nunca devem passar pelo browser.
	
	\subsection*{4.3}
	Indirectamente sim. No entanto, depois de um utilizador se autenticar junto do servidor de autorização, é gerada uma resposta com código 302 (redirect), para o browser com o callback da aplicação cliente no header location e com CODE. o browser gera um pedido get para o callback da aplicação cliente, que recebe o CODE, e o envia para o token endpoint do servidor de autorização junto com o client id e o client secret. Em resposta, o servidor de autorização envia o ID Token, que, de acordo com o parâmetro scope, inclui a informação de identidade a que utilizador deu acesso à aplicação cliente.

\section{Exercício 5}
	O ID Token é obtido quando o relying party acede ao token endpoint, onde troca CODE pelo ID Token.
	É um objecto estruturado, assinado e cifrado e serve ao relying party para garantir que o utilizador se autenticou junto do itentity provider, contendo informação de perfil, que pode ser usada para personalizar a experiência do utilizador.
	Este também pode ser usado pelo relying party para associar um utilizador ao seu autenticador, sendo uma má prática publicá-lo directamente no autenticador a entregar ao utilizador.

\section{Exercício 6}
	i) 	Primeiro, efectuou-se a autenticação do servidor, usando o browser como aplicação cliente:
	\begin{enumerate}
		\item Foi gerado o ficheiro PEM com chave privada do servidor 'secure-server-private.pem' através do comando: openssl pkcs12 -in secure-server.pfx -out secure-server-private.pem -nodes
		\item Foi gerado o ficheiro PEM com certificado do servidor 'secure-server.pem' através do comando : openssl x509 -inform der -in secure-server.cer -out secure-server.pem
		\item Foi adicionado ao ficheiro \path{C:\Windows\System32\drivers\etc\hosts} a linha '127.0.0.1 www.secure-server.edu'
		\item Foram comentadas na aplicação servidora as linhas
		\begin{lstlisting}
		ca: fs.readFileSync('CA1.pem'), 
		requestCert: true, 
		rejectUnauthorized: true 
		\end{lstlisting}
		\item Acedeu-se a \url{https:\\www.secure-server.edu:4433}
	\end{enumerate}
	Em seguida, efectuou-se, ainda usando o browser como aplicação cliente, a autenticação de cliente da seguinte forma:
	\begin{enumerate}
		\item Gerou-se o ficheiro com o certificado PEM da CA root através do comando openssl x509 -inform der -in CA1.cer -out CA1.pem
		\item Instalaram-se os quatro certificados-folha com o Assistente para importar certificados do sistema operativo.
		\item Acedeu-se a \url{https:\\www.secure-server.edu:4433} e obteve-se a resposta 200 OK, com a string 'Secure Hello World with node.js'.
	\end{enumerate}

	ii) Foi desenvolvida a aplicação cliente Java constante de HttpsClient.java.
	Foram gerados os KeyStores (.jks) com a chave privada do cliente (client.jks) e com os certificados das raízes de confiança (trustedroots.jks). Para gerar o primeiro ficheiro, executaram-se os comandos:
	\begin{enumerate}
	\item keytool -importkeystore -srckeystore Alice\textunderscore1.pfx -srcstoretype pkcs12 -destkeystore client.jks -deststoretype JKS
	
	\item keytool -changealias -alias ``1`` -destalias ``Alice\textunderscore 1`` -keystore 
	``client.jks``
	\end{enumerate}
	Para gerar o segundo ficheiro, executaram-se os comandos:
	\begin{enumerate}
		\item keytool -importcert -file ``CA1.cer`` -keystore trustedroots.jks -alias ``CA1``
		\item keytool -importcert -file ``CA1-int.cer`` -keystore trustedroots.jks -alias ``CA1-int``
	\end{enumerate}

	iii) Foi testada a aplicação desenvolvida, com e sem autenticação de cliente. Para desabilitar autenticação de cliente foram comentadas na aplicação cliente as linhas
	\begin{lstlisting}
	ca: fs.readFileSync('CA1.pem'), 
	requestCert: true, 
	rejectUnauthorized: true 
	\end{lstlisting}
	Na aplicação cliente, foi desabilitado o carregamento do KeyStore com o certificado do cliente. 
\newpage


\end{document}
